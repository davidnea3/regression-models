\documentclass[]{article}
\usepackage{lmodern}
\usepackage{amssymb,amsmath}
\usepackage{ifxetex,ifluatex}
\usepackage{fixltx2e} % provides \textsubscript
\ifnum 0\ifxetex 1\fi\ifluatex 1\fi=0 % if pdftex
  \usepackage[T1]{fontenc}
  \usepackage[utf8]{inputenc}
\else % if luatex or xelatex
  \ifxetex
    \usepackage{mathspec}
  \else
    \usepackage{fontspec}
  \fi
  \defaultfontfeatures{Ligatures=TeX,Scale=MatchLowercase}
\fi
% use upquote if available, for straight quotes in verbatim environments
\IfFileExists{upquote.sty}{\usepackage{upquote}}{}
% use microtype if available
\IfFileExists{microtype.sty}{%
\usepackage{microtype}
\UseMicrotypeSet[protrusion]{basicmath} % disable protrusion for tt fonts
}{}
\usepackage[margin=1in]{geometry}
\usepackage{hyperref}
\hypersetup{unicode=true,
            pdftitle={Regression models},
            pdfauthor={Davyd},
            pdfborder={0 0 0},
            breaklinks=true}
\urlstyle{same}  % don't use monospace font for urls
\usepackage{color}
\usepackage{fancyvrb}
\newcommand{\VerbBar}{|}
\newcommand{\VERB}{\Verb[commandchars=\\\{\}]}
\DefineVerbatimEnvironment{Highlighting}{Verbatim}{commandchars=\\\{\}}
% Add ',fontsize=\small' for more characters per line
\usepackage{framed}
\definecolor{shadecolor}{RGB}{248,248,248}
\newenvironment{Shaded}{\begin{snugshade}}{\end{snugshade}}
\newcommand{\KeywordTok}[1]{\textcolor[rgb]{0.13,0.29,0.53}{\textbf{#1}}}
\newcommand{\DataTypeTok}[1]{\textcolor[rgb]{0.13,0.29,0.53}{#1}}
\newcommand{\DecValTok}[1]{\textcolor[rgb]{0.00,0.00,0.81}{#1}}
\newcommand{\BaseNTok}[1]{\textcolor[rgb]{0.00,0.00,0.81}{#1}}
\newcommand{\FloatTok}[1]{\textcolor[rgb]{0.00,0.00,0.81}{#1}}
\newcommand{\ConstantTok}[1]{\textcolor[rgb]{0.00,0.00,0.00}{#1}}
\newcommand{\CharTok}[1]{\textcolor[rgb]{0.31,0.60,0.02}{#1}}
\newcommand{\SpecialCharTok}[1]{\textcolor[rgb]{0.00,0.00,0.00}{#1}}
\newcommand{\StringTok}[1]{\textcolor[rgb]{0.31,0.60,0.02}{#1}}
\newcommand{\VerbatimStringTok}[1]{\textcolor[rgb]{0.31,0.60,0.02}{#1}}
\newcommand{\SpecialStringTok}[1]{\textcolor[rgb]{0.31,0.60,0.02}{#1}}
\newcommand{\ImportTok}[1]{#1}
\newcommand{\CommentTok}[1]{\textcolor[rgb]{0.56,0.35,0.01}{\textit{#1}}}
\newcommand{\DocumentationTok}[1]{\textcolor[rgb]{0.56,0.35,0.01}{\textbf{\textit{#1}}}}
\newcommand{\AnnotationTok}[1]{\textcolor[rgb]{0.56,0.35,0.01}{\textbf{\textit{#1}}}}
\newcommand{\CommentVarTok}[1]{\textcolor[rgb]{0.56,0.35,0.01}{\textbf{\textit{#1}}}}
\newcommand{\OtherTok}[1]{\textcolor[rgb]{0.56,0.35,0.01}{#1}}
\newcommand{\FunctionTok}[1]{\textcolor[rgb]{0.00,0.00,0.00}{#1}}
\newcommand{\VariableTok}[1]{\textcolor[rgb]{0.00,0.00,0.00}{#1}}
\newcommand{\ControlFlowTok}[1]{\textcolor[rgb]{0.13,0.29,0.53}{\textbf{#1}}}
\newcommand{\OperatorTok}[1]{\textcolor[rgb]{0.81,0.36,0.00}{\textbf{#1}}}
\newcommand{\BuiltInTok}[1]{#1}
\newcommand{\ExtensionTok}[1]{#1}
\newcommand{\PreprocessorTok}[1]{\textcolor[rgb]{0.56,0.35,0.01}{\textit{#1}}}
\newcommand{\AttributeTok}[1]{\textcolor[rgb]{0.77,0.63,0.00}{#1}}
\newcommand{\RegionMarkerTok}[1]{#1}
\newcommand{\InformationTok}[1]{\textcolor[rgb]{0.56,0.35,0.01}{\textbf{\textit{#1}}}}
\newcommand{\WarningTok}[1]{\textcolor[rgb]{0.56,0.35,0.01}{\textbf{\textit{#1}}}}
\newcommand{\AlertTok}[1]{\textcolor[rgb]{0.94,0.16,0.16}{#1}}
\newcommand{\ErrorTok}[1]{\textcolor[rgb]{0.64,0.00,0.00}{\textbf{#1}}}
\newcommand{\NormalTok}[1]{#1}
\usepackage{graphicx,grffile}
\makeatletter
\def\maxwidth{\ifdim\Gin@nat@width>\linewidth\linewidth\else\Gin@nat@width\fi}
\def\maxheight{\ifdim\Gin@nat@height>\textheight\textheight\else\Gin@nat@height\fi}
\makeatother
% Scale images if necessary, so that they will not overflow the page
% margins by default, and it is still possible to overwrite the defaults
% using explicit options in \includegraphics[width, height, ...]{}
\setkeys{Gin}{width=\maxwidth,height=\maxheight,keepaspectratio}
\IfFileExists{parskip.sty}{%
\usepackage{parskip}
}{% else
\setlength{\parindent}{0pt}
\setlength{\parskip}{6pt plus 2pt minus 1pt}
}
\setlength{\emergencystretch}{3em}  % prevent overfull lines
\providecommand{\tightlist}{%
  \setlength{\itemsep}{0pt}\setlength{\parskip}{0pt}}
\setcounter{secnumdepth}{0}
% Redefines (sub)paragraphs to behave more like sections
\ifx\paragraph\undefined\else
\let\oldparagraph\paragraph
\renewcommand{\paragraph}[1]{\oldparagraph{#1}\mbox{}}
\fi
\ifx\subparagraph\undefined\else
\let\oldsubparagraph\subparagraph
\renewcommand{\subparagraph}[1]{\oldsubparagraph{#1}\mbox{}}
\fi

%%% Use protect on footnotes to avoid problems with footnotes in titles
\let\rmarkdownfootnote\footnote%
\def\footnote{\protect\rmarkdownfootnote}

%%% Change title format to be more compact
\usepackage{titling}

% Create subtitle command for use in maketitle
\newcommand{\subtitle}[1]{
  \posttitle{
    \begin{center}\large#1\end{center}
    }
}

\setlength{\droptitle}{-2em}
  \title{Regression models}
  \pretitle{\vspace{\droptitle}\centering\huge}
  \posttitle{\par}
  \author{Davyd}
  \preauthor{\centering\large\emph}
  \postauthor{\par}
  \predate{\centering\large\emph}
  \postdate{\par}
  \date{May, 2018}


\begin{document}
\maketitle

\subsection{Executive Summary}\label{executive-summary}

You work for Motor Trend, a magazine about the automobile industry.
Looking at a data set of a collection of cars, they are interested in
exploring the relationship between a set of variables and miles per
gallon (MPG) (outcome). They are particularly interested in the
following two questions:

``Is an automatic or manual transmission better for MPG'' ``Quantify the
MPG difference between automatic and manual transmissions''

\subsection{Load Data}\label{load-data}

Load required packages, dataset, and convert categorical variables to
factors.

\begin{Shaded}
\begin{Highlighting}[]
\KeywordTok{library}\NormalTok{(ggplot2)}
\end{Highlighting}
\end{Shaded}

\begin{verbatim}
## Warning: package 'ggplot2' was built under R version 3.4.4
\end{verbatim}

\begin{Shaded}
\begin{Highlighting}[]
\KeywordTok{data}\NormalTok{(mtcars)}
\KeywordTok{head}\NormalTok{(mtcars, }\DataTypeTok{n=}\DecValTok{3}\NormalTok{)}
\KeywordTok{dim}\NormalTok{(mtcars)}
\NormalTok{mtcars}\OperatorTok{$}\NormalTok{cyl <-}\StringTok{ }\KeywordTok{as.factor}\NormalTok{(mtcars}\OperatorTok{$}\NormalTok{cyl)}
\NormalTok{mtcars}\OperatorTok{$}\NormalTok{vs <-}\StringTok{ }\KeywordTok{as.factor}\NormalTok{(mtcars}\OperatorTok{$}\NormalTok{vs)}
\NormalTok{mtcars}\OperatorTok{$}\NormalTok{am <-}\StringTok{ }\KeywordTok{factor}\NormalTok{(mtcars}\OperatorTok{$}\NormalTok{am)}
\NormalTok{mtcars}\OperatorTok{$}\NormalTok{gear <-}\StringTok{ }\KeywordTok{factor}\NormalTok{(mtcars}\OperatorTok{$}\NormalTok{gear)}
\NormalTok{mtcars}\OperatorTok{$}\NormalTok{carb <-}\StringTok{ }\KeywordTok{factor}\NormalTok{(mtcars}\OperatorTok{$}\NormalTok{carb)}
\KeywordTok{attach}\NormalTok{(mtcars)}
\end{Highlighting}
\end{Shaded}

\subsection{Exploratory Data Analysis}\label{exploratory-data-analysis}

\textbf{See Appendix I} Box plot comparing MPG between Automatic and
Manual transmission. The results show a significant increase in MPG for
manual transmission when compared to automatic transmission.

\subsubsection{Statistical Inference}\label{statistical-inference}

T-Test between transmission type and MPG

\begin{Shaded}
\begin{Highlighting}[]
\NormalTok{testResults <-}\StringTok{ }\KeywordTok{t.test}\NormalTok{(mpg }\OperatorTok{~}\StringTok{ }\NormalTok{am)}
\NormalTok{testResults}\OperatorTok{$}\NormalTok{p.value}
\end{Highlighting}
\end{Shaded}

\begin{verbatim}
## [1] 0.001373638
\end{verbatim}

The T-Test rejects the null hypothesis that there is no difference in
MPG for both transmission types.

\begin{Shaded}
\begin{Highlighting}[]
\NormalTok{testResults}\OperatorTok{$}\NormalTok{estimate}
\end{Highlighting}
\end{Shaded}

\begin{verbatim}
## mean in group 0 mean in group 1 
##        17.14737        24.39231
\end{verbatim}

The estimated difference between the two transmission types is 7.24494
MPG in favour of manual transmission.

\subsubsection{Regression Analysis}\label{regression-analysis}

Fitting the model for the data

\begin{Shaded}
\begin{Highlighting}[]
\NormalTok{fullModelFit <-}\StringTok{ }\KeywordTok{lm}\NormalTok{(mpg }\OperatorTok{~}\StringTok{ }\NormalTok{., }\DataTypeTok{data =}\NormalTok{ mtcars)}
\KeywordTok{summary}\NormalTok{(fullModelFit)  }\CommentTok{# results hidden}
\KeywordTok{summary}\NormalTok{(fullModelFit)}\OperatorTok{$}\NormalTok{coeff  }\CommentTok{# results hidden}
\end{Highlighting}
\end{Shaded}

Since none of the p-values are below 0.05, we cannot conlude that there
is any statistical significance.

Selecting variables which are most statistically significant

\begin{Shaded}
\begin{Highlighting}[]
\NormalTok{stepFit <-}\StringTok{ }\KeywordTok{step}\NormalTok{(fullModelFit)}
\KeywordTok{summary}\NormalTok{(stepFit) }\CommentTok{# results hidden}
\KeywordTok{summary}\NormalTok{(stepFit)}\OperatorTok{$}\NormalTok{coeff }\CommentTok{# results hidden}
\end{Highlighting}
\end{Shaded}

The new model has 4 variables (cylinders, horsepower, weight,
transmission). The R-squared value of 0.8659 confirms that this model
explains about 87\% of the variance in MPG. The p-values also are
statistically significantly because they have a p-value less than 0.05.
The coefficients conclude that increasing the number of cylinders from 4
to 6 with decrease the MPG by 3.03. Further increasing the cylinders to
8 with decrease the MPG by 2.16. Increasing the horsepower is decreases
MPG 3.21 for every 100 horsepower. Weight decreases the MPG by 2.5 for
each 1000 lbs increase. A Manual transmission improves the MPG by 1.81.

\subsection{Residuals \& Diagnostics}\label{residuals-diagnostics}

Residual Plot

\textbf{See Appendix Figure II}

The plots show that:

\begin{enumerate}
\def\labelenumi{\arabic{enumi}.}
\tightlist
\item
  The randomness of the Residuals vs.~Fitted plot supports the
  assumption of independence
\item
  The points of the Normal Q-Q plot following closely to the line
  conclude that the distribution of residuals is normal
\item
  The Scale-Location plot random distribution confirms the constant
  variance assumption
\item
  Since all points are within the 0.05 lines, the Residuals vs.~Leverage
  concludes that there are no outliers
\end{enumerate}

\begin{Shaded}
\begin{Highlighting}[]
\KeywordTok{sum}\NormalTok{((}\KeywordTok{abs}\NormalTok{(}\KeywordTok{dfbetas}\NormalTok{(stepFit)))}\OperatorTok{>}\DecValTok{1}\NormalTok{)}
\end{Highlighting}
\end{Shaded}

\begin{verbatim}
## [1] 0
\end{verbatim}

\subsection{Conclusion}\label{conclusion}

There is a difference in MPG between transmission type. A manual
transmission will have a slight advantage in MPG. However, weight,
horsepower, \& number of cylinders are more statistically significant
when determining MPG.

\subsection{Appendix I}\label{appendix-i}

\includegraphics{Regression_Models_Project_files/figure-latex/unnamed-chunk-7-1.pdf}

\subsection{Appendix II}\label{appendix-ii}

\includegraphics{Regression_Models_Project_files/figure-latex/unnamed-chunk-8-1.pdf}


\end{document}
